http://mathb.in/51301

//http://mathb.in/51298


Rotate vector V to V'
---
###given
 $V$ is a vector representing a point in space to be rotated around an axis

($A$, $a$) defining the axis and angle to rota- te; this is also a definition of a frame...

###result
 $V'$ is a vector rotated by the scaled frame.
 


---

- $ V' =  \cos(a) V + \sin( a )( V \times A) + (1-\cos( a )) ( A \cdot V ) A $

### notes:
 - To apply inverse multiply $a$ by $-1$.  The axis might also be multiplied by -1; the result is the same, computationally it's fewer operations to just modify the angle.
 - to apply a partial timestep scale $a$ by $dT$


---
## Rotate A around B

###given
A frame defined by ($A$, $a$) as the Axis and Angle, or  $(x,y,z)$ unit vector, and $\theta$. 

Another frame ($B$, $b$) as the Axis and Angle  $(x,y,z)$ unit vector, and $\theta$.

###result 

($R'$,$\theta$) that is the composite frame rotating A around B.

---

- $  \theta = 2 \cos^{-1} ( \cos\frac b 2\cos\frac a 2-\sin\frac b 2\sin\frac a 2 ( A \cdot B ) )  $
  
- $   R = (\sin\frac b 2\cos\frac b 2)B 
           + (\sin\frac a 2\cos\frac b 2)A 
           + (\sin\frac a 2\sin\frac b 2)( A \times B )$

- $  R' = \frac R {||R||} $  

or

- $\theta = 2 \cos^{-1} ( \frac { (1 - A \cdot B) \cos( \frac {a - b} 2) + (1 + A \cdot B) \cos(\frac {a + b} 2) } 2 ) $


      
- $  R = (\sin \frac {a + b} 2 - \sin \frac {a - b} 2) B 
           + ( \sin\frac {a - b} 2 + \sin\frac {a + b} 2) A 
           + ( \cos\frac {a - b} 2 - \cos\frac {a + b} 2)( A � B ) )$

  - this version of the axis computation has a divide by $2$, but since it's being  normalized anyway, can skip the multiplication by a constant

- $  R' = \frac R {||R||} $  

### notes:
 - To apply inverse multiply $a$ by $-1$ 
 - to apply a partial timestep scale $a$ by $dT$





---

# Interpolation

### Given
- A frame $Q$ as $(Q,q)$
- Another frame $P$ and $(P,p)$
- A scalar $D$ from $0$ to $1$ of the orientation between the two frames to compute.

### result

- $Q'$ as a step 

## Operations

- Apply `Rotate A around B` to rotate $P$ around the $Q$ inversed, result with ($T$, $t$);
- Apply `Rotate vector V to V' ` to rotate $T$ around $Q$, result with $T'$
- use `Rotate A around B` to rotate $Q$ around ($T'$,$t \cdot D$).  The angle may have a scalar applied between 0,1.  Result with $Q'$  as $(Q',q')$


# Integration

(to be reveiwed)
---

- `Q.freeSpin(P) // update the frame Q by rotating around P`
- `newV = Q.apply(V,dt) // rotates vector V into frame scaled by dT(for example +/-1)  `

Angular Velocity is considered as an axis-angle frame rotation.

$R = [x,y,z]$ as a quaternion $\cos(\frac {||R||} 2) + sin(\frac {||R||} 2)( xi + yj + zk )$

$A = [a,b,c]$ as a quaternion $ 0 + ai + bj + ck$

$A \times Q$ or $Q \times A$ results as a vector which is used as angle-axis, or rotation quaternion

## External Force 

- `R_0 = R.freeSpin( R.apply( A, -1 ) );`
- `R_N = R_[N-1].freeSpin( R_[N-1].apply( A, -1 ) );`


$R_0 = R \times( A \times R ) )$

$R_N = R_{(N-1)} \times(  A_ \times R_{(N-1)} ) )$

## Internal force

- `R_0 = R.freeSpin( R.apply( A, 1 ) );`
- `R_N = R_[N-1].freeSpin( R_[N-1].apply( A, 1 ) );`

$R_0 = R \times( R \times A ) )$

$R_N = R_{(N-1)} \times(  R_{(N-1)} \times A ) )$



## Rodrigues' Rotation Formula Composite 

given 

- ${Q} = [n,\theta]$ n is a normal vector/axis of rotation
- ${P} = [n,\theta]$ $\theta$ is the angle of rotation - often positive.

- 

 - $ A = Q_n \cdot P_n $
 - $ dA = d(A \cdot B) = 1 \cdot B dA + A \cdot 1 dB $

 - $ B = \cos \frac {{ Q_\theta} + P_{\theta}} 2 $  
 - $ dB = d(cos(a + b)) = -sin(a + b) da - sin(a + b) db$

 - $ C = \cos \frac {{ Q_\theta} - P_{\theta}} 2 $
 - $ dC = d(cos(a - b)) = -sin(a - b) da + sin(a - b) db $

 - $ D = \frac { C( 1 - A )  + B(1+A) } 2 $
 - $ dD = d(1/2 ((1 + A) B + C(1 - A))) = 1/2 (B - C'(1 - A)) dA + 1/2 (1 + A) dB $

 - $ {Result}_{\theta} = 2 \arccos( D ) $

- 
 - $ E = \sin \frac {{ Q_\theta} + P_{\theta}} 2 $
 - $ dE = d(sin(a + b)) = cos(a + b) da + cos(a + b) db $

 - $ F = \sin \frac {{ Q_\theta} - P_{\theta}} 2 $
 - $ dF = d(sin(a - b)) = cos(a - b) da - cos(a - b) db $

 - $ G = ( Q_n \times P_n ) (C-B) + P_n (E+F) + Q_n (E-F) $
 - $ d(A \cross B) = 1\cross B dA + A\cross 1 dB $

 - $ {Result}_n = \frac G  {||G||} $
 

$ {Result} = Result_{\theta} {Result}_n  $


